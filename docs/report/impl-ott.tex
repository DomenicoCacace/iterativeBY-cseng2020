\section{Implementazione ed Ottimizzazione}\label{impl}
\subsection*{\textbf{Rappresentazione dei polinomi}}
I polinomi, elementi dell'anello $GF(2)[x]$, sono salvati in forma binaria compatta: ogni bit in un byte rappresenta un 
coefficiente di un polinomio binario in formato Big-Endian (leggendo da sinistra a destra, i primi elementi che
si incontrano sono i coefficienti di grado più alto)

Una \textit{machine word} $A_i$ è considerata un \texttt{DIGIT} ed i byte che
compongono un \texttt{DIGIT} sono in formato Big Endian; allo stesso modo, 
anche gli elementi con più \texttt{DIGIT} sono rappresentati in 
formato Big Endian.
\begin{table}
    \begin{tabular}{lll}
    Bin & Hex & Polinomio \\
    \hline
    0000 0000 & 0x00 & 0 \\
    0000 0001 & 0x01 & 1 \\
    0000 0002 & 0x02 & $x$ \\
    0000 0003 & 0x03 & $x+1$ \\
    $\cdots$ & $\cdots$ & $\cdots$ \\
    0000 1111 & 0x0F & $x^3+x^2+x+1$ \\
    $\cdots$ & $\cdots$ & $\cdots$ \\
    1111 1111 & 0xFF & $x^7+x^6+x^5+\cdots+x+1$ \\
    \end{tabular}
    \caption{Rappresentazione dei polinomi, singolo byte}
\end{table}
\subsubsection*{Estensioni dell'ISA x86}
L'implementazione proposta in~\cite{benchmark} utilizza AVX, un'estensione 
dell' Instruction Set Architecture di x86 che permette di lavorare su più parole di memoria
in parallelo, portando notevoli vantaggi nelle operazioni aritmetiche: ciò significa,
su una macchina a 64 bit (dimensione di un \texttt{DIGIT}), lavorare su 128 o 256 bit.
Questo ci permette inoltre di avere un'implementazione molto efficiente per i prodotti, utilizzando funzioni
altamente specializzate per polinomi di 8 \texttt{DIGIT} o meno e loro combinazioni per dimensioni
maggiori~\cite{bodrato2007towards}; gli intrinsics permettono anche l'implementazione delle diverse versioni di
\texttt{divstep}.



\subsection*{\textbf{Punti critici}}\label{crit}
Analizzando l'esecuzione dell'algoritmo tramite \texttt{callgrind}, notiamo che la 
maggior parte del tempo di esecuzione è dovuta a \texttt{gf2x\_scalarprod}: ciò non ci sorprende, in quanto
questa funzione è invocata due volte per il calcolo di $f^{\prime}(x)$ e $g^{\prime}(x)$ (righe 7-8), e altre quattro per $\mathcal{P \times \mathcal{Q}}$ (riga 10); inoltre,
per quest'ultima operazione, sono necessarie altrettante chiamate a memcpy, per spostare i risultati ottenuti da un vettore temporaneo alla
destinazione desiderata.

La funzione \texttt{gf2x\_scalarprod} prende in input due coppie di polinomi ($a_{0},a_{1}$, $b_{0},b_{1}$), di dimensione rispettivamente $na$ e $nb$ \texttt{DIGIT}, e 
produce in output il prodotto scalare $res = a_{0} \cdot b_{0} + a_{1} \cdot b_{1}$, di dimensione $nr = na+nb$; in realtà, per poter eseguire le moltiplicazioni
tra polinomi, è necessario spostare l'operando più corto in un buffer e aggiungere zeri nelle posizioni dei coefficienti più significativi mancanti, fino
a raggiungere la dimensione dell'altro operando. 

Una volta calcolati i prodotti, questi vengono ulteriormente manipolati, troncando i coefficienti superflui, shiftandoli (calcolo di $f^{\prime}$ e $g^{\prime}$, righe 7-8) e 
spostandoli da una posizione all'altra; analizzando come questi polinomi vengono manipolati, possiamo andare a salvare i risultati direttamente nelle posizioni
corrette, riduendo i tempi di esecuzione.

\subsection*{\textbf{Ottimizzazioni}}\label{ott}

\subsubsection*{Allocazione dei polinomi}
Il primo passo nell'unrolling dell'algoritmo ricorsivo consiste nell'allocare in una volta sola tutta la memoria necessaria per salvare
i risultati e gli operandi intermedi, ossia $ \mathcal{P}, \mathcal{Q}, f^{\prime}$ e $g^{\prime}$; le stesse porzioni di memoria vengono riutilizzate da diverse
chiamate in momenti diversi, per cui viene presa la dimensione massima per ogni \textit{slot}, individuabile percorrendo il ramo destro dell'albero di ricorsione
dalla radice all'ultima foglia. Abbiamo quindi, con un albero di ricorsione di profondità $d$:
\begin{itemize}
        \item $\mathcal{P}$: 4 array composti da $d$ componenti, ognuno di dimensione \texttt{num\_digits\_j};
        \item $\mathcal{Q}$: 4 array composti da $d$ componenti, ognuno di dimensione \texttt{num\_digits\_nminusj};
        \item $f^{\prime}$: 1 array composto da $d-1$ componenti, ognuno di dimensione \texttt{num\_digits\_n} + \texttt{num\_digits\_j};
        \item $g^{\prime}$: come al punto precedente.
\end{itemize}
Tenendo traccia delle dimensioni dei vari componenti è poi possibile determinare gli indici di accesso agli array per i vari valori di $d$.

\subsubsection*{Scomposizione dei prodotti}
Come accenato, le funzioni per il calcolo del prodotto tra polinomi richiedono operandi di dimensione uguale: invece di aggiungere coefficienti
nulli al polinomio più corto~($a$), possiamo dividere il polinomio più lungo ($b$) in due parti: la parte \textit{inferiore} ($b_{0}$, coefficienti meno significativi),
della stessa dimensione di $a$, viene moltiplicata direttamente con $a$ stesso; il procedimento viene quindi ripetuto considerando ora il polinomio $a$ e 
la parte restante dell'altro polinomio, $b_{1}$, fino ad esaurimento.
I vari sottoprodotti che vengono calcolati vanno quindi sommati: dati due sottoprodotti successivi $r_i$, $r_{i+1}$, abbiamo che $\deg{r_{i+1}}-\deg{r_i} = na$,
(in modo simile a quanto sopra, $na$ indica la dimensione del poliniomio più corto alla $i$-esima iterazione).

Quindi, nell'implementazione del generatore, sostituiamo ogni chiamata a \texttt{gf2x\_scalarprod} con la corrispondente sequenza di \texttt{gf2x\_add} e \texttt{GF2X\_MUL},
utilizzando un accumulatore, $nr$, per tenere traccia dell'offset da considerare;
inoltre, nel caso i due operandi siano di dimensioni inferiori agli 8 \texttt{DIGIT}, andiamo ad utilizzare direttamente le funzioni specifiche per la dimensione
desiderata (\texttt{gf2x\_mul\_X\_avx}), evitando le chiamate a \texttt{GF2X\_MUL}.

\subsubsection*{Troncamento e shift dei risultati}
In quasi tutti i casi l'output di \text{gf2x\_scalarprod} viene troncato, eliminando sempre una porzione dei coefficienti più significativi; dato che il
metodo di moltiplicazione parte dai coefficienti meno significativi, possiamo bloccarne l'esecuzione prematuramente appena abbiamo tutti i coefficienti
necessari, evitando il calcolo di quelli che sarebbero poi tagliati. Ciò viene banalmente implementato  nel generatore utilizzando lo stesso accumulatore
$nr$ citato al punto prima, ma sottraendovi il numero di \texttt{DIGIT} che sarebbero eliminati successivamente. 

A questo punto, possiamo quindi evitare di utilizzare un vettore ausiliario per il calcolo dei prodotti (righe 6-8, 10), ma ne abbiamo ancora bisogno
per il calcolo di ${(f^{\prime}}\div{x^{j}})$ e $({g^{\prime}}\div{x^{j}})$: queste divisioni sono infatti ottenute tramite uno shift a sinistra di $j$ bit; per risolvere
questo problema, dividiamo lo shift in due parti:
\begin{itemize}
    \item shift di ($\lfloor{j}\div{\texttt{DIGIT\_SIZE\_b}}$\footnote{Dimensione di un \texttt{DIGIT} in bit}$\rfloor$) \texttt{DIGIT} interi;
    \item shift di ($j \mod \texttt{DIGIT\_SIZE\_b}$) bit singoli.
\end{itemize}
Lo shift dei \texttt{DIGIT} interi può essere ricondotto al problema del troncamento dei coefficienti più significativi, aggiustando opportunamente gli offset; per quanto riguarda
lo shift dei singoli bit (che saranno per costruzione meno di quelli in un \texttt{DIGIT}), abbiamo bisogno di aggiungere un \texttt{DIGIT} in più per ognuno dei polinomi
$f^{\prime}$ e $g^{\prime}$, in modo da poter shiftare i bit del polinomio desiderato senza andare a sovrascrivere il \texttt{DIGIT} meno significativo del polinomio precedente.
