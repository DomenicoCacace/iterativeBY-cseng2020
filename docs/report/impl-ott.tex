\section{Implementazione ed Ottimizzazione}\label{impl}
\subsection*{\textbf{Rappresentazione dei polinomi}}
I polinomi, elementi dell'anello $GF(2)[x]$, sono salvati in forma binaria compatta: ogni bit in un byte rappresenta un 
coefficiente di un polinomio binario in formato Big-Endian (leggendo da sinistra a destra, i primi elementi che
si incontrano sono i coefficienti di grado più alto)

Una \textit{machine word} è considerata un \texttt{digit} ed i byte che
compongono un \texttt{digit} sono in formato Big Endian; allo stesso modo, 
anche gli elementi con più \texttt{digit} sono rappresentati in 
formato Big Endian.
\begin{table}
    \caption{Rappresentazione dei polinomi, singolo byte}
    \begin{tabular}{lll}
    \hline
    \textbf{Bin} & \textbf{Hex} & \textbf{Polinomio} \\
    \hline
    0000 0000 & 0x00 & 0 \\
    0000 0001 & 0x01 & 1 \\
    0000 0002 & 0x02 & $x$ \\
    0000 0003 & 0x03 & $x+1$ \\
    $\cdots$ & $\cdots$ & $\cdots$ \\
    0000 1111 & 0x0F & $x^3+x^2+x+1$ \\
    $\cdots$ & $\cdots$ & $\cdots$ \\
    1111 1111 & 0xFF & $x^7+x^6+x^5+\cdots+x+1$ \\
    \hline
    \end{tabular}
\end{table}
\subsubsection*{Estensioni dell'ISA x86}
L'implementazione proposta in~\cite{benchmark} utilizza AVX, un'estensione 
dell' Instruction Set Architecture di x86 che permette di lavorare su più parole di memoria
in parallelo, portando notevoli vantaggi nelle operazioni aritmetiche: ciò significa,
su una macchina a 64 bit (dimensione di un \texttt{digit}), lavorare su 128 o 256 bit.
Questo ci permette inoltre di avere un'implementazione molto efficiente per i prodotti, utilizzando funzioni
altamente specializzate per polinomi di 8 \texttt{digit} o meno e loro combinazioni per dimensioni
maggiori, come riportato in~\cite{bodrato2007towards}; gli intrinsics, funzioni gestite direttamente dal compilatore che permettono di usare
istruzioni specifiche dell'architettura per cui si sta compilando come fossero chiamate a funzioni standard, permettono inoltre 
l'implementazione delle diverse versioni di \texttt{divstep}.



\subsection*{\textbf{Punti critici}}\label{crit}
Tramite un'analisi di complessità asintotica, supportata dai risultati sperimentali ottenuti tramite \texttt{callgrind}, notiamo che la 
maggior parte del tempo di esecuzione è dovuta a \texttt{gf2x\_scalarprod}: ciò non ci sorprende, in quanto
questa funzione è invocata due volte per il calcolo di $\bar{f(x)}$ e $\bar{g(x)}$ (\texttt{jumpdivstep}, righe 7-8), e altre quattro per $P \times Q$ (\texttt{jumpdivstep}, riga 10); inoltre,
per quest'ultima operazione, sono necessarie altrettante chiamate a \texttt{memcpy}, per spostare i risultati ottenuti da un vettore temporaneo alla
destinazione desiderata.
\begin{algorithm}
    \small
    \SetAlgoLined
    \SetKwFunction{scalarprod}{scalarprod}    
    \KwIn{$a_0, a_1, b_0, b_1$: array di polinomi\newline
                $na, nb$: dimensioni degli array di polinomi}
    \KwOut{$res = a_0\cdot b_0 + a_1 \cdot b_1$, dimensione $na+nb$}
    \BlankLine
    \SetKwProg{Fn}{function}{:}{}

    \Fn{\scalarprod{$na, a_0, a_1, nb, b_0, b_1$}}{

        \texttt{DIGIT} tmp[$na*2$]\;
        \texttt{DIGIT} tmp2[$na*2$]\;
        \texttt{DIGIT} buffer[$na$]\;
        \texttt{memset}(buffer, 0x00, ($na-nb$)*\texttt{DIGIT\_SIZE\_B})\;
        \BlankLine
        \texttt{memcpy}(buffer+($na-nb$), $b_0$, $nb$*\texttt{DIGIT\_SIZE\_B})\;
        \texttt{GF2X\_MUL}($na*2$, tmp, $a_0$, buffer)\;
        \BlankLine
        \texttt{memcpy}(buffer+($na-nb$), $b_1$, $nb$*\texttt{DIGIT\_SIZE\_B})\;
        \texttt{GF2X\_MUL}($na*2$, tmp2, $a_1$, buffer)\;
        \BlankLine
        \texttt{gf2x\_add}($na*2$, tmp, tmp, tmp2)\;
        \texttt{memcpy}(res, tmp+($na-nb$), ($na+nb$)*\texttt{DIGIT\_SIZE\_B})\;
        \Return{res}
    }
    \caption{\texttt{gf2x\_scalarprod} (per $na > nb$)}
    \end{algorithm}

La funzione \texttt{gf2x\_scalarprod} prende in input due coppie di polinomi ($a_{0},a_{1}$, $b_{0},b_{1}$), di dimensione rispettivamente $na$ e $nb$ \texttt{digit}, e 
produce in output il prodotto scalare $res = a_{0} \cdot b_{0} + a_{1} \cdot b_{1}$, di dimensione $nr = na+nb$; in realtà, per poter eseguire le moltiplicazioni
tra polinomi, è necessario spostare l'operando più corto in un buffer (righe 6, 8) e aggiungere zeri nelle posizioni dei coefficienti più significativi mancanti (riga 5), fino
a raggiungere la dimensione dell'altro operando. 

Una volta calcolati i prodotti, questi vengono ulteriormente manipolati, troncando i coefficienti superflui, shiftandoli (\texttt{jumpdivstep}, righe 7-8) e 
spostandoli da una posizione all'altra; analizzando come questi polinomi vengono manipolati, possiamo andare a salvare i risultati direttamente nelle posizioni
corrette, riduendo i tempi di esecuzione.

\subsection*{\textbf{Ottimizzazioni}}\label{ott}

\subsubsection*{Allocazione dei polinomi}
Il primo passo nell'unrolling dell'algoritmo ricorsivo consiste nell'allocare in una volta sola tutta la memoria necessaria per salvare
i risultati e gli operandi intermedi, ossia $ P, Q, \bar{f(x)}$ e $\bar{g(x)}$; le stesse porzioni di memoria vengono riutilizzate da diverse
chiamate in momenti diversi, per cui viene presa la dimensione massima per ogni \textit{slot}, individuabile percorrendo il ramo destro dell'albero di ricorsione
dalla radice all'ultima foglia. Partendo con operandi di grado $n_0-1$, \texttt{digit} di dimensione $s$ e parole di memoria di dimensione $ws$
(entrambe espresse in bit), avremo un albero di ricorsione di profondità $d=\lceil \log_{2}\frac{n}{s} \rceil$; per ogni livello, avremo 
\begin{equation*}
    j_{i} = \floor*{\left(\floor*{\frac{n_i}{2}} + ws - 2\right) \cdot \frac{1}{ws-1}} \cdot (ws-1)
\end{equation*}
\begin{equation*}
    n_{i} = j_{i-1} \qquad (i \neq 0)
\end{equation*}
Le operazioni di somma, prodotto e divisione che includono $ws$ sono necessarie ad ottenere, alla fine della ricorsione, polinomi di grado $ws-1$\footnote{In alcuni casi 
i polinomi ottenuti sono di grado inferiore per \textit{mancanza di coefficienti}}, in modo da sfruttare appieno gli intrinsics di AVX.

Partendo da quanto appena trovato, possiamo calcolare il numero di \texttt{digit} necessario a memorizzare i vari polinomi:
\begin{equation*}
    \begin{aligned}
    \mathrm{\texttt{num\_digits\_n}_i} = \ceil*{\frac{n_i}{s}}\\
    \mathrm{\texttt{num\_digits\_j}_i} = \ceil*{\frac{j_i}{s}}\\
    \mathrm{\texttt{num\_digits\_nminusj}_i} = \ceil*{\frac{n_i - j_i}{s}}\\
    \end{aligned}
\end{equation*}

Possiamo quindi individuare le dimensioni degli slot dei vari array in base al livello di profondità $i$:
\begin{itemize}
        \item $P$: $d$ slot di dimensione $\mathrm{\texttt{num\_digits\_j}}_i$, 4 array
        \item $P$: $d$ slot di dimensione $\mathrm{\texttt{num\_digits\_nminusj}}_i$, 4 array
        \item $\bar{f(x)}$: $d-1$ slot di dimensione $\mathrm{\texttt{num\_digits\_n}}_i + \mathrm{\texttt{num\_digits\_j}}_i$, 1 array
        \item $\bar{g(x)}$: come al punto precedente.
\end{itemize}
Tenendo traccia delle dimensioni dei vari componenti è poi possibile determinare gli indici di accesso agli array per i vari valori di $d$.

\subsubsection*{Scomposizione dei prodotti}

Come accenato, le funzioni per il calcolo del prodotto tra polinomi richiedono operandi di dimensione uguale: invece di aggiungere coefficienti
nulli al polinomio più corto~($a = 0, a_1$), possiamo dividere il polinomio più lungo ($b = b_0, b_1$) in due parti: 
\begin{center}
    \begin{tabular}{cccc}
            & $0$ & $a_1$ & $\times$ \\
            & $b_0$ & $b_1$ & =\\
            \hline
            & $0\cdot b_1$ &$a_1\cdot b_1$ & +\\
            &\hspace{-9ex}$0\cdot b_0$&\hspace{-9ex}$a_1\cdot b_0$&\hspace{-10ex}\--
    \end{tabular}
\end{center}
la parte \textit{inferiore} ($b_{1}$, coefficienti meno significativi),
della stessa dimensione di $a_1$, viene moltiplicata direttamente con $a_1$ stesso; il procedimento viene quindi ripetuto considerando ora il polinomio $a_1$ e 
la parte restante dell'altro polinomio, $b_{0}$, fino ad esaurimento.


I vari sottoprodotti che vengono calcolati vanno quindi sommati: dati due sottoprodotti successivi $r_i$, $r_{i+1}$, abbiamo che $\deg({r_{i+1}})-\deg({r_i}) = na$
(in modo simile a quanto sopra, $na$ indica la dimensione del poliniomio più corto alla $i$-esima iterazione): con riferimento all'esempio, abbiamo che $r_1 = a_1\cdot b_1$ e $r_2 = a_1\cdot b_0$,
per cui i $na$ coefficienti meno significativi di $r_1$ vengono sommati ai $na$ coefficienti più significativi di $r_2$, mentre i restanti ($na$ per $r_1$, $nb$ per $r_2$) vengono semplicemente riportati 
come sono nel risultato, in quanto i sottoprodotti restanti sono nulli.


Quindi, nell'implementazione del generatore, sostituiamo ogni chiamata a \texttt{gf2x\_scalarprod} con la corrispondente sequenza di \texttt{gf2x\_add} e \texttt{GF2X\_MUL},
utilizzando un accumulatore, $nr$, per tenere traccia dell'offset da considerare;
inoltre, nel caso i due operandi siano di dimensioni inferiori agli 8 \texttt{digit}, andiamo ad utilizzare direttamente le funzioni specifiche per la dimensione
desiderata (\texttt{gf2x\_mul\_X\_avx}), evitando le chiamate a \texttt{GF2X\_MUL}.

\subsubsection*{Troncamento e shift dei risultati}
In quasi tutti i casi l'output di \texttt{gf2x\_scalarprod} viene troncato, eliminando sempre una porzione dei coefficienti più significativi che eccedono la dimensione necessaria; dato che il
metodo di moltiplicazione parte dai coefficienti meno significativi, possiamo bloccarne l'esecuzione prematuramente appena abbiamo tutti i coefficienti
necessari, evitando il calcolo di quelli che sarebbero poi tagliati. Ciò viene banalmente implementato  nel generatore utilizzando lo stesso accumulatore
$nr$ citato al punto prima, ma sottraendovi il numero di \texttt{digit} che sarebbero eliminati successivamente. 

A questo punto, possiamo quindi evitare di utilizzare un vettore ausiliario per il calcolo dei prodotti (\texttt{jumpdivstep}, righe 6-8, 10), ma ne abbiamo ancora bisogno
per il calcolo di $\frac{\bar{f(x)}}{x^{j}}$ e $\frac{\bar{g(x)}}{x^{j}}$: queste divisioni sono infatti ottenute tramite uno shift a sinistra di $j$ bit; per risolvere
questo problema, dividiamo lo shift in due parti:
\begin{itemize}
    \item shift di ($\lfloor\frac{j}{\texttt{DIGIT\_SIZE\_b}}$\footnote{Dimensione di un \texttt{digit} in bit}$\rfloor$) \texttt{digit} interi;
    \item shift di ($j \mod \texttt{DIGIT\_SIZE\_b}$) bit singoli.
\end{itemize}
Lo shift dei \texttt{digit} interi può essere ricondotto al problema del troncamento dei coefficienti più significativi, aggiustando opportunamente gli offset; per quanto riguarda
lo shift dei singoli bit (che saranno per costruzione meno di quelli in un \texttt{digit}), abbiamo bisogno di aggiungere un \texttt{digit} in più per ognuno dei polinomi
$\bar{f(x)}$ e $\bar{g(x)}$, in modo da poter shiftare i bit del polinomio desiderato senza andare a sovrascrivere il \texttt{digit} meno significativo del polinomio precedente.
