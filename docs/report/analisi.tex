\section{Analisi dell'algoritmo}\label{analisi}
In ~\cite{bernstein2019fast} gli autori descrivono due metodi per il calcolo di massimo comun divisore e inverso moltiplicativo: il primo per numeri interi,
il secondo per anelli di polinomi; nello specifico, andremo a sfruttare quest'ultimo per anelli di polinomi a coefficienti binari per le ragioni precedentemente citate.

L'algoritmo di inversione che andiamo ad analizzare è basato su un approccio \textit{divide et impera}: il polinomio da invertire e il modulo vengono suddivisi dalla funzione
\texttt{jumpdivstep}, che prende in input due polinomi $f(x), g(x)$, di grado al più $n=2p-1$, con $\delta = \deg{(f(x))} - \deg{(g(x))}$ e individua
un \textit{pivot} per suddividere gli operandi in due metà\footnote{Questa scelta si è rivelata essere ottimale, come riportato in ~\cite{barenghi2020comprehensive}, ma 
qualsiasi valore di $j$ compreso tra $1$ ed $n-1$ è valido}, ovvero $j = \lfloor \frac{n}{2} \rfloor$. 

La funzione \texttt{jumpdivstep} è quindi chiamata ricorsivamente passando come parametri le due 
metà dei polinomi appena individuate, finché la loro dimensione non è sufficientemente piccola (ovvero fin quando la dimensione non è pari o inferiore a quella di una \textit{machine word}):
a questo punto viene invocata la funzione \texttt{divstep}, che gestisce il caso base della ricorsione;
i risultati trovati vengono quindi ricombinati fino ad ottenere il risultato finale.

Nello specifico, la funzione \texttt{invert} chiama la funzione \texttt{jumpdivstep} passando come parametri le rappresentazioni riflesse del modulo $f(x)$ e dell'elemento da invertire $a(x)$: per rappresentazione riflessa
intendiamo il polinomio ottenuto invertendo il coefficiente del termine $x^i$ con quello del termine $x^{\deg(S(x))-i}$, ottenuto tramite la funzione \texttt{mirror}; alla fine, la funzione
ritorna la rappresentazione riflessa dell'inverso di $a(x)$, $H_{0,1}$, ottenuta dal prodotto dei risultati parziali calcolati dalle chiamate ricorsive.

 \begin{algorithm}
    \small
    \SetAlgoLined
    \SetKwFunction{jumpdivstep}{jumpdivstep}
    \SetKwFunction{invert}{invert}
    \KwIn{$f(x), g(x)$: polinomi binari\newline
                $n$: grado massimo di $f(x)$\newline
                $\delta = \deg{f(x)} - \deg{g(x)}$}
    \KwOut{$P\times Q$: matrice di polinomi, risultato parziale dell'inversione}
    \KwData{$ws$: dimensione in bit di una \textit{machine word}\newline
                    $P$, $Q$: matrici di polinomi, contengono i risultati parziali delle due chiamate ricorsive}
    \BlankLine

    \SetKwFunction{divstep}{divstep}
    \SetKwProg{Fn}{function}{:}{}
    \Fn{\jumpdivstep{$n, \delta, f(x), g(x)$}}{

    \If{$n \leq ws$}{
        \Return\texttt{divstep($n, \delta, f(x), g(x)$)\;}
    }

        $j \leftarrow \lfloor \frac{n}{2} \rfloor$\;
        $\delta, P \leftarrow$ \texttt{jumpdivstep}($j, \delta, f(x)\mod x^j, g(x)\mod x^j$)\;
        $\bar{f(x)} \leftarrow P_{0,0}\cdot f(x) + P_{0,1}\cdot g(x)$\;
        $\bar{g(x)} \leftarrow P_{1,0}\cdot f(x) + P_{1,1}\cdot g(x)$\;
        $\delta, Q \leftarrow$ \texttt{jumpdivstep}($n-j, \delta, \frac{\bar{f(x)}}{x^j}, \frac{\bar{g(x)}}{x^j}$)\;

        \Return{$\delta, (P\times Q)$}\;
    }
    \caption{\texttt{jumpdivstep}}
    \end{algorithm}

    \begin{algorithm}
        \small
        \SetAlgoLined
        \KwIn{$f(x)$: polinomio irriducibile di $GF(2^p)$\newline
                    $a(x)$: elemento invertibile di $GF(2^p)$}
        \KwOut{$V(x)$, con $a^{-1}(x) \equiv V(x) \in GF(2^p)$}
        \SetKwFunction{invert}{invert}
    
        \SetKwFunction{divstep}{divstep}
        \SetKwProg{Fn}{function}{:}{}    
        \BlankLine
        \Fn{\invert{$f(x), a(x)$}}{
            $S(x) \leftarrow$ \texttt{mirror}($f(x)$)\;
            $R(x) \leftarrow$ \texttt{mirror}($a(x)$)\;
            $\delta, H \leftarrow$ \texttt{jumpdivstep}($2p - 1, 1, S(x), R(x)$)\;
            $V(x) \leftarrow$ \texttt{mirror}($H_{0,1}$)\;
            \Return$V(x)$\;
    
        }
        \caption{\texttt{invert}}
    \end{algorithm}



    \subsection*{\textbf{\textit{Simulazione} dell'algoritmo}} Possiamo, partendo dall'algoritmo qui presentato, ricostruire un albero di ricorsione binario tracciando tutte le
    chiamate ed i vari parametri necessari: la prima chiamata a \texttt{jumpdivstep} (\texttt{invert}, riga 4) è la radice dell'albero, 
    la prima chiamata effettuata (\texttt{jumpdivstep}, riga 6) corrisponde al figlio destro di un nodo, mentre la seconda (\texttt{jumpdivstep}, riga 9) corrisponde al
    figlio sinistro; quando la dimensione degli operandi è pari o inferiore alla \textit{word size}, ossia quando viene 
    invocata la funzione \texttt{divstep}, il nodo corrispondente nell'albero di ricorsione risulta essere una foglia.
    
    Analizzando l'albero di ricorsione ed i parametri relativi ai nodi, possiamo inoltre determinare le dimensioni
    dei vettori che contengono i risultati intermedi ($P, Q$) e gli operandi intermedi
    ($\bar{f(x)}, \bar{g(x)}$), nonché gli indici di accesso ai suddetti per ogni chiamata. Da qui, possiamo \textit{simulare}
    le chiamate attraversando l'albero usando un approccio \textit{depth-first}, dando priorità ai figli destri
    (che per costruzione corrispondono alla prima chiamata).
    È inoltre interessante notare che le operazioni ed i parametri necessari per ogni chiamata dipendano solo dalla 
    dimensione dei parametri di input, e non dai parametri stessi: dato che le dimensioni sono note tramite un'analisi statica, possiamo derivare
    l'albero di ricorsione a compile time; questo ci permetterà di effettuare tutte le ottimizzazioni
    riportate in~\ref{impl}, e quindi poter generare una versione
    iterativa di \texttt{jumpdivstep} specifica per dimensione dell'input.

    \subsubsection*{Contenuto dei nodi}
    Ad ogni nodo dell'albero di ricorsione corrisponde una chiamata a \texttt{jumpdivstep}, ognuna caratterizzata
    dai propri parametri, tra cui:
    \begin{itemize}
        \item \texttt{n}: grado massimo dei operandi ricevuti in input
        \item \texttt{j}: grado massimo degli operandi passati alla prima chiamata ricorsiva (figlio destro)
        \item \texttt{n-j}: grado massimo degli operandi passati alla seconda chiamata ricorsiva (figlio sinistro)
        \item \texttt{num\_digits\_(n|j|nminusj)}: dimensione degli operandi di cui sopra, espressa in \texttt{digit}
    \end{itemize}


